\documentclass[]{article}

\usepackage{layouts}
\usepackage{tikz}

%for environment aligned
\usepackage{amsmath}

\usepackage[active,tightpage]{preview}
\PreviewEnvironment{tikzpicture}

\usetikzlibrary{calc}

\begin{document}



%note: align= is needed to print nodes with several lines
\tikzstyle{every node}= [font=\Huge, align=center]
%\tikzstyle{every node}= [font=\Huge, draw=black,                     minimum width=2cm, thick]

\begin{tikzpicture}
	
\pgfmathparse{28.4527}
\global\let\tikzconst\pgfmathresult
	
%\node at (1,4) {textwidth in cm: \printinunitsof{cm}\prntlen{\textwidth}};
\node (v13) at (1,7) {Multilines \\ test};

\node(v11) at (1,2) {};
\path
  let
    \p1=(v11)  
  in {
  node at (v11)
	  {
	  	\pgfmathparse{(\x1/\tikzconst)}
	  	\global\let\xpos\pgfmathresult
	  	\pgfmathparse{(\y1/\tikzconst)}
	  	\global\let\ypos\pgfmathresult
	  	This node is at \pgfmathprintnumber{\xpos}; \pgfmathprintnumber{\ypos}
	  }
  };


\node(v1) at (1,0) {$x=\pgfmathprintnumber{\xpos}; y=\pgfmathprintnumber{\ypos}$};

\node[left] (v3) at (1,-2) {$\frac{x+y}{2}=$};
    
\node[right] at (1,-2) {
		\pgfmathparse{0.5* ( \ypos + \xpos)}
  		\global\let\tempvar\pgfmathresult
		$\tempvar pt$
	};



\node at (5,-9) {Example:} 
  child {node {
  $\begin{aligned} \\ 
    \frac{x+y}{d}&=\pi\\
     a+b &= d +1
  \end{aligned}$}};
   
\node[] at (-2,-9) {Example: \\ 
  $\begin{aligned}
     a &= bx + c\\
     a+b &= d +1
  \end{aligned}$};





\end{tikzpicture}

\end{document}